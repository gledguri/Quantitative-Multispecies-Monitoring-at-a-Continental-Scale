\documentclass[9pt,twoside,lineno]{pnas-new}
% Use the lineno option to display guide line numbers if required.

\templatetype{pnassupportinginfo}

\title{Quantitative, Multispecies Monitoring at a Continental Scale}
\author{Gledis Guri, Owen Liu, Ryan P. Kelly, Megan Shaffer, Kim Parsons, Krista M. Nichols, Pedro F. P. Brandão-Dias and Andrew Olaf Shelton}
\correspondingauthor{Gledis Guri.\\E-mail: gguri@uw.edu}

\begin{document}

\maketitle

%% Adds the main heading for the SI text. Comment out this line if you do not have any supporting information text.
\SItext

\section*{Extended Methods}
\subsection*{qPCR}
Environmental samples were analysed in triplicate using the multiplexed assay targeting Pacific hake, Eulachon and Pacific lamprey as described in \cite{ramon-laca2021} on a QuanStudio 6 (Applied Biosystems). Only Pacific hake was used in this study due to the remaining targets having little positive amplification. The samples underwent real-time thermocycler protocol including an initial denaturation step at 95°C 10 min followed by 40 cycles of 15 s at 95°C and 1 min at 58°C. All thermocycler reactions were run in 20 $\mu$L volume consisting of 10 $\mu$L of TaqMan Environmental Master Mix 2.0, 1 $\mu$L of each primer (forward: AAATGTTTAAACTAGAGCCGAATAGC and reverse: TCGTGGAGTCAAAGTGGGGTAGA, 10 nM concentration each), 0.5 $\mu$L of probe (AAATGTTTAAACTAGAGCCGAATAGC; 10nM concentration), 3 $\mu$L of dH20 and 2 $\mu$L of DNA template. To monitor quality control, all runs included an internal positive control (IPC) to detect PCR inhibition and a negative template control (NTC) containing water instead of DNA template. Any IPC delay exceeding 0.5 cycles in the NTC was considered inhibition.

Alongside environmental samples standard samples were analyzed constructed from a 120 bp synthetic DNA fragment (gBlock; IDT) representing the 12S region of Pacific hake (\textit{Merluccius productus}), which encompasses the 101 bp 12S qPCR target (sequence information available in Ramón-Laca et al. 2021). The synthetic DNA fragment was diluted to create a series of standards with final concentrations ranging from $10^{0}$ to $10^{5}$ copies/$\mu$L.

\subsection*{Metabarcoding}
In total, 568 samples, including 554 environmental samples, 7 PCR blanks and 7 positive controls (holding only kangaroo DNA) were amplified were amplified using MiFish-U universal primers \cite{miya2015} with Illumina tails (forward TCGTCGGCAGCGTCAGATGTGTATAAGAGACAGGCCGGTAAAACTCGTGCCAGC; reverse GTCTCGTGGGCTCGGAGATGTGTATAAGAGACAGCATAGTGGGGTATCTAATCCCAGTTTG) by using a two-step PCR protocol. The DNA was amplified in the first PCR reaction (PCR 1) in a 20 µL reaction consisting of: 10 µL of Phusion Master Mix (2X), 0.4 µL of the forward primer (10 µM); 0.4 µL of the reverse primer (10 µM), 0.6 µL of 100\% DMSO, 0.5 µL of rAlbumin (20 µg/µL), 4.4 µL of nuclease-free water and 2 µL of DNA template. Reactions were run with the following cycling conditions: an initial denaturation of 98°C for 30 sec; followed by 35 cycles of 98°C for 10 sec, 60°C for 30 sec, and of 72°C for 3 sec; with a final extension of 72°C for 10 min. 

PCR product was cleaned using Ampure Beads (1.2x) and then indexed in a second PCR reaction (PCR 2) consiting of 8 cycles. Resulting products were visualized on a 2\% agarose gel and quantified using Quant-iT™ dsDNA Assay Kit (Thermo Fisher Scientific, USA) with Fluoroskan™ Microplate Fluorometer (Thermo Fisher Scientific, USA). Indexed products were pooled into libraries for sequencing, and then size- selected to extract only the target fish band using the E-Gel™ SizeSelect™ II Agarose Gels (Thermo Fisher Scientific, USA).  Subsequently the libraries were sequenced on Illumina MiSeq platform using the v3 600 cycle kit. 

%A total of 554 environmental samples of metabarcoding samples sequenced are 554 of which 535 had higher sequencing depth than 1000 reads (indicating good quality samples). From those we indicated that only 371 samples had Pacific hake (\textit{Merluccius productus}) present jointly with qPCR positive detection of the same species. All libraries produced a total of 56,279,345 reads with 54,285,713 reads (96.5\%) belonging to class \textit{Actinopteri}. A total of 222 unique species was identified from which 165 belonged to fish. 

\subsection*{Mock community}

We quantified the total genomic DNA in each extract using Qubit HS assay. We then quantified the concentration of the 12S rRNA gene using MarVer1 primers (forward: CGTGCCAGCCACCGCG; reverse: GGGTATCTAATCCYAGTTTG \cite{valsecchi2020}), which perfectly match the template and give unbiased estimates of concentration, using ddPCR (Bio-Rad, Inc., QX200 Droplet Digital PCR system). Each tissue of each species was quantified in a 22 $\mu$L reaction consisted of 2 $\mu$L of DNA template from genomic DNA , 11 $\mu$L of ddPCR EvaGreen (Bio-Rad), 0.22 $\mu$L of each forward and reverse primers (10 uM), and 0.56 $\mu$L of nuclease free water. The thermocycler reactions were run in C1000 Touch Thermal Cycler with 96-Deep Well Reaction Module (Bio-Rad) using the PCR program as follows: 2 min at 50°C for enzyme activation, 2 min at 95°C for initial denaturation, and 40 cycles of denaturation for 1 sec at 95°C and primer annealing and elongation for 30 sec min at 60°C, with a ramp rate of 2°C per s and heldat 4°C until droplets were read. Droplets were determined to be positive after drawing a threshold based on NTCs.

\subsection*{Bioinformatics}
PEDRO

\subsection*{Bayesian model}
OLE

\subsection*{Model convergance}
OLE + GLED


%%% Each figure should be on its own page
\begin{figure}
\centering
\includegraphics[width=0.99\textwidth]{plots/5. Supplementary Figure 1.jpg}
\caption{First figure}
\end{figure}

%%% Each figure should be on its own page
\begin{figure}
\centering
\includegraphics[width=0.99\textwidth]{plots/6. Supplementary Figure 2.jpg}
\caption{First figure}
\end{figure}

%%% Each figure should be on its own page
\begin{figure}
\centering
\includegraphics[width=0.99\textwidth]{plots/7. Supplementary Figure 3.jpg}
\caption{First figure}
\end{figure}

%%% Each figure should be on its own page
\begin{figure}
\centering
\includegraphics[width=0.99\textwidth]{plots/8. Supplementary Figure 4.jpg}
\caption{First figure}
\end{figure}

%%% Each figure should be on its own page
\begin{figure}
\centering
\includegraphics[width=0.80\textwidth]{plots/9. Supplementary Figure 5.jpg}
\caption{First figure}
\end{figure}

%%% Each figure should be on its own page
\begin{figure}
\centering
\includegraphics[width=0.89\textwidth]{plots/10. Supplementary Figure 6.jpg}
\caption{First figure}
\end{figure}

\begin{table}\centering
\caption{Initial relative abundances of the selected species across different mock communities.}
    \begin{tabular}{lcccccccc}
        \toprule
        {Species} & \multicolumn{2}{c}{Mock1} & \multicolumn{2}{c}{Mock2} & \multicolumn{2}{c}{Mock3} & \multicolumn{2}{c}{Mock4} \\
        & Even & Skew & Even & Skew & Even & Skew & Even & Skew \\
        \midrule
        \textit{Clupea pallasii}           & 0.437 & 0.325  & 0.042  & 0.009     & NA         & NA          & 0.025      & 0.038 \\
        \textit{Engraulis mordax}          & 0.122 & 0.034  & 0.15 & 0.15       & 0.146      & 0.16       & 0.308      & 0.478 \\
        \textit{Leuroglossus stilbius}     & 0.108 & 0.29   & NA & NA         & NA         & NA         & NA         & NA \\    
        \textit{Merluccius productus}      & 0.079 & 0.144  & 0.12 & 0.06       & 0.117      & 0.075      & 0.247      & 0.287 \\
        \textit{Microstomus pacificus}     & NA    & NA     & 0.057  & 0.215     & NA         & NA         & NA         & NA \\    
        \textit{Sardinops sagax}           & 0.028 & 0.033  & 0.14 & 0.035      & 0.137      & 0.025      & 0.289      & 0.131 \\
        \textit{Scomber japonicus}         & NA    & NA     & 0.122  & 0.053      & 0.119      & 0.027     & NA         & NA \\    
        \textit{Sebastes entomelas}        & NA    & NA     & NA & NA          & 0.068      & 0.218     & NA         & NA \\    
        \textit{Stenobrachius leucopsarus} & NA    & NA     & NA & NA          & 0.356      & 0.487     & NA         & NA \\    
        \textit{Tactostoma macropus}       & NA    & NA     & 0.13 & 0.324     & NA         & NA         & NA         & NA \\    
        \textit{Tarletonbeania crenularis} & NA    & NA     & 0.083  & 0.124     & NA         & NA         & NA         & NA \\    
        \textit{Thaleichthys pacificus}    & 0.123 & 0.113  & 0.099  & 0.012     & NA         & NA          & 0.013      & 0.021 \\
        \textit{Trachurus symmetricus}     & 0.103 & 0.06   & 0.057  & 0.018      & 0.056      & 0.008      & 0.118      & 0.04 \\
        \bottomrule
    \end{tabular}
\end{table}

\bibliography{pnas-sample}

\end{document}
